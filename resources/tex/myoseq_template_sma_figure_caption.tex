The y-axis denotes the probability that an individual is a carrier for SMA, and the X axis shows the proportion of SMN reads that aligned to SMN1 (SMN1 reads/ SMN1 + SMN2 reads). The dotted black line at 0.38 marks the SMN1 read proportion typical of a carrier (0.33) allowing for a 5\% type 1 error. Individuals with both extremes of the 95\% credible interval below 0.38 are considered likely carriers, those with an interval spanning but not completely below 0.38 are classified as possible carriers, and those with intervals not spanning 0.38 are unlikely carriers. 
\\ \\
Individuals with a carrier probability close to 1.0 and proportion of SMN1 reads close to zero may be affected individuals with homozygous deletions of SMN1. These individuals could benefit from SMA treatment and should have further testing in a CLIA certified lab to confirm the diagnosis. Homozygous deletions of SMN1 are typically validated using Multiplex Ligation-dependent Probe Amplification (MLPA). This test can also be used to analyze copy number variation in SMN2 that results from recombination events between SMN1 and SMN2. The SMN2 gene has an exon skipping rate of 50-90\% and can therefore produce functional SMN protein at a lower rate than SMN1. The severity of SMA in diagnosed patients is often determined by SMN2 copy number, with high SMN2 copy numbers in resulting in less severe phenotypes and later onset than low SMN2 copy numbers.

